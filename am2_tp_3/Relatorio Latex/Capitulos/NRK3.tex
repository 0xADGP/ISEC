% Método de Runge-Kutta de Ordem 3
%--------------------------------------
\chapter{Método de Runge-Kutta de Ordem 3}

\section*{6.1 Fórmulas}

O Método de Runge-Kutta de Ordem 3 (RK3) fornece uma aproximação mais precisa em relação ao RK2, utilizando três avaliações da função por iteração.

\textbf{Fórmula Geral:}
\begin{equation}
y_{i+1} = y_i + \frac{1}{6}(k_1 + 4k_2 + k_3)
\end{equation}

\textbf{Cálculo de $k_1$:}
\begin{equation}
k_1 = h f(t_i, y_i)
\end{equation}

\textbf{Cálculo de $k_2$:}
\begin{equation}
k_2 = h f\left(t_i + \frac{h}{2}, y_i + \frac{k_1}{2}\right)
\end{equation}

\textbf{Cálculo de $k_3$:}
\begin{equation}
k_3 = h f(t_i + h, y_i - k_1 + 2k_2)
\end{equation}

onde:
\begin{itemize}
    \item $y_{i+1}$ → Próximo valor aproximado da solução do problema original;
    \item $y_i$ → Valor aproximado da solução na abscissa atual;
    \item $h$ → Passo de integração;
    \item $f(t_i, y_i)$ → Avaliação da função diferencial;
    \item $k_1$, $k_2$, $k_3$ → Aproximações das inclinações.
\end{itemize}

\section*{6.2 Algoritmo/Função}

\textbf{Algoritmo:}
\begin{enumerate}
    \item Definir o passo $h$;
    \item Criar um vetor $y$ para armazenar os valores aproximados;
    \item Atribuir o valor inicial $y(1) = y_0$;
    \item Calcular $k_1$ com os valores atuais;
    \item Calcular $k_2$ no ponto médio do intervalo;
    \item Calcular $k_3$ com base em $k_1$ e $k_2$;
    \item Calcular $y_{i+1}$ usando a média ponderada das inclinações.
\end{enumerate}
