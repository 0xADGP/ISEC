
% Conclusão
%--------------------------------------
\chapter{Conclusão}

A realização desta atividade permitiu aprofundar os conhecimentos sobre métodos numéricos para a resolução de Equações Diferenciais Ordinárias (EDO), com especial foco nos métodos de Euler, Euler melhorado, Runge-Kutta de ordens 2, 3 e 4, e o método \texttt{ode45} do \textsc{Matlab}.

Foi possível compreender não só a fundamentação teórica destes métodos, mas também as suas implementações práticas, as diferenças em termos de precisão, estabilidade e custo computacional, bem como as situações em que cada um se revela mais vantajoso. 

Através da comparação entre os resultados obtidos por diferentes métodos, ficou evidente a importância de se considerar o equilíbrio entre a complexidade do método e a precisão desejada, bem como o controlo do erro associado à discretização.

Conclui-se que os métodos de Runge-Kutta, em especial o de quarta ordem e o método \texttt{ode45}, se destacam pela sua elevada precisão em problemas com soluções suaves, sendo adequados para a maioria das aplicações práticas. No entanto, métodos mais simples como o de Euler continuam a ser úteis em contextos onde a simplicidade e o baixo custo computacional são prioritários.

Em suma, esta atividade contribuiu significativamente para a consolidação de competências na área da análise numérica, nomeadamente na aplicação prática de técnicas fundamentais para a simulação de sistemas dinâmicos descritos por EDOs.
