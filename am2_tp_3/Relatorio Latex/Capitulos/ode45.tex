% Função ode45 do MATLAB
%--------------------------------------
\chapter{Função \texttt{ode45} do MATLAB}

\section*{8.1 Descrição}

A função \texttt{ode45} do MATLAB é um dos métodos numéricos mais utilizados para resolver equações diferenciais ordinárias (EDOs). Ela é baseada no método de Runge-Kutta de ordem 4(5), ou seja, uma combinação de dois métodos de Runge-Kutta de ordens 4 e 5, que permite controle adaptativo do passo de integração.

A função é adequada para a maioria dos problemas não-stiff e fornece uma solução com alta precisão de forma eficiente.

\section*{8.2 Sintaxe}

\begin{verbatim}
[t, y] = ode45(@f, [t0 tf], y0)
\end{verbatim}

onde:
\begin{itemize}
    \item \texttt{@f} → Handle da função que define a EDO, ou seja, $y' = f(t, y)$;
    \item \texttt{[t0 tf]} → Intervalo de integração, com tempo inicial $t_0$ e final $t_f$;
    \item \texttt{y0} → Condição inicial da variável dependente $y$;
    \item \texttt{t} → Vetor de tempos nos quais a solução foi avaliada;
    \item \texttt{y} → Solução numérica aproximada da EDO nos tempos definidos em \texttt{t}.
\end{itemize}

\section*{8.3 Exemplo de Uso}

\textbf{Exemplo: Resolver a equação diferencial $y' = -2y$, com $y(0) = 1$, de $t = 0$ até $t = 5$.}

\begin{verbatim}
f = @(t, y) -2*y;
[t, y] = ode45(f, [0 5], 1);
plot(t, y)
xlabel('t')
ylabel('y(t)')
title('Solução da EDO usando ode45')
\end{verbatim}

\section*{8.4 Vantagens}

\begin{itemize}
    \item Método adaptativo: ajusta automaticamente o passo para maior precisão;
    \item Simples de usar com apenas a definição da função e condições iniciais;
    \item Ideal para problemas com solução suave e bem comportada.
\end{itemize}
