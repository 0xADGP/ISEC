% Método de Runge-Kutta de Ordem 2
%--------------------------------------
\chapter{Método de Runge-Kutta de Ordem 2}

\section*{5.1 Fórmulas}

O Método de Runge-Kutta de Ordem 2 (RK2) é um método de passo simples que requer apenas derivadas de primeira ordem e pode fornecer aproximações precisas para a solução de Problemas de Valor Inicial (PVI).

\textbf{Fórmula Geral:}
\begin{equation}
y_{i+1} = y_i + \frac{1}{2}(k_1 + k_2)
\end{equation}

\textbf{Cálculo de $k_1$:}
\begin{equation}
k_1 = h f(t_i, y_i)
\end{equation}

\textbf{Cálculo de $k_2$:}
\begin{equation}
k_2 = h f(t_i + h, y_i + k_1)
\end{equation}

onde:
\begin{itemize}
    \item $y_{i+1}$ → Próximo valor aproximado da solução do problema original (na abscissa $t_{i+1}$);
    \item $y_i$ → Valor aproximado da solução do problema original na abscissa atual;
    \item $h$ → Valor de cada subintervalo (passo);
    \item $f(t_i , y_i)$ → Valor da equação diferencial avaliada em $t_i$ e $y_i$;
    \item $k_1$ → Inclinação no início do intervalo;
    \item $k_2$ → Inclinação no fim do intervalo.
\end{itemize}

\section*{5.2 Algoritmo/Função}

\textbf{Algoritmo:}
\begin{enumerate}
    \item Definir o passo $h$;
    \item Criar um vetor $y$ para guardar a solução;
    \item Atribuir o primeiro valor de $y$ (condição inicial do PVI);
    \item Calcular a inclinação no início do intervalo ($k_1$);
    \item Calcular a inclinação no fim do intervalo ($k_2$);
    \item Calcular a média das inclinações;
    \item Calcular o valor aproximado para a $i$-ésima iteração usando o método de RK2.
\end{enumerate}
