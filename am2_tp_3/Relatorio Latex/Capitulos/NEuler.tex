% Método de Euler
%--------------------------------------
\chapter{Método de Euler}

\section*{3.1 Fórmulas}

O método de Euler é um procedimento numérico de primeira ordem ($y'$) para aproximar a solução da equação diferencial que satisfaz a condição inicial:

\begin{equation}
y' = f(t, y)
\end{equation}

\begin{equation}
y(t_0) = y_0
\end{equation}

O Método de Euler para resolver um Problema de Valor Inicial (PVI) é dado pela seguinte fórmula geral:

\begin{equation}
y_{i+1} = y_i + h f(t_i, y_i)
\end{equation}

onde:
\begin{itemize}
    \item $y_{i+1}$ → Próximo valor aproximado da solução do problema original (na abscissa $t_{i+1}$);
    \item $y_i$ → Valor aproximado da solução do problema original na abscissa atual;
    \item $h$ → Valor de cada subintervalo (passo);
    \item $f(t_i , y_i)$ → Valor da equação em $t_i$ e $y_i$.
\end{itemize}

\section*{3.2 Algoritmo/Função}

\textbf{Algoritmo:}
\begin{enumerate}
    \item Definir o valor do passo $h$;
    \item Criar um vetor $y$ para guardar a solução e atribuir $y(1) = y_0$;
    \item Atribuir o primeiro valor de $y$;
    \item Para $i$ de $1$ a $n$, calcular o método de Euler para a $i$-ésima iteração.
\end{enumerate}
