% Método de Runge-Kutta de Ordem 4
%--------------------------------------
\chapter{Método de Runge-Kutta de Ordem 4}

\section*{7.1 Fórmulas}

O Método de Runge-Kutta de Ordem 4 (RK4) é um dos métodos mais utilizados para resolver equações diferenciais ordinárias devido à sua alta precisão e simplicidade de implementação.

\textbf{Fórmula Geral:}
\begin{equation}
y_{i+1} = y_i + \frac{1}{6}(k_1 + 2k_2 + 2k_3 + k_4)
\end{equation}

\textbf{Cálculo de $k_1$:}
\begin{equation}
k_1 = h f(t_i, y_i)
\end{equation}

\textbf{Cálculo de $k_2$:}
\begin{equation}
k_2 = h f\left(t_i + \frac{h}{2}, y_i + \frac{k_1}{2}\right)
\end{equation}

\textbf{Cálculo de $k_3$:}
\begin{equation}
k_3 = h f\left(t_i + \frac{h}{2}, y_i + \frac{k_2}{2}\right)
\end{equation}

\textbf{Cálculo de $k_4$:}
\begin{equation}
k_4 = h f(t_i + h, y_i + k_3)
\end{equation}

onde:
\begin{itemize}
    \item $y_{i+1}$ → Próximo valor aproximado;
    \item $y_i$ → Valor atual;
    \item $h$ → Tamanho do passo;
    \item $f(t_i, y_i)$ → Equação diferencial;
    \item $k_1$, $k_2$, $k_3$, $k_4$ → Avaliações intermediárias.
\end{itemize}

\section*{7.2 Algoritmo/Função}

\textbf{Algoritmo:}
\begin{enumerate}
    \item Definir o passo $h$;
    \item Criar um vetor $y$ para armazenar a solução;
    \item Atribuir o valor inicial $y(1) = y_0$;
    \item Calcular $k_1$ com os valores atuais;
    \item Calcular $k_2$ com base em $k_1$;
    \item Calcular $k_3$ com base em $k_2$;
    \item Calcular $k_4$ com base em $k_3$;
    \item Calcular $y_{i+1}$ usando a média ponderada das quatro inclinações.
\end{enumerate}
